
\documentclass[12pt]{article}
\usepackage[utf8]{inputenc}
\usepackage[T1]{fontenc}
\usepackage[french]{babel}
\usepackage{amsmath, amssymb}
\usepackage{geometry}
\usepackage{hyperref}
\usepackage{enumitem}
\usepackage{array}
\usepackage{longtable}
\geometry{margin=2.5cm}

\title{Contribution à l’amélioration de la rétention des apprenants par l’intelligence artificielle : État de l’art}
\author{Doudou COLY}
\date{\today}

\begin{document}
\maketitle

\section{Introduction}
Dans un contexte de transformation numérique accélérée de l’enseignement supérieur, la problématique de la rétention des apprenants dans les dispositifs en ligne s’impose comme une préoccupation majeure. Malgré l’accessibilité élargie offerte par les plateformes numériques, les taux d’abandon restent élevés, dépassant fréquemment les 70\% selon des rapports internationaux. Cette étude vise à explorer les potentialités offertes par l’intelligence artificielle pour améliorer cette rétention, en croisant les apports issus du domaine éducatif et d’autres secteurs ayant déjà recours à ces technologies de manière avancée.

\textbf{Question de recherche.} \emph{Comment les solutions d’intelligence artificielle, validées dans d’autres secteurs, peuvent-elles être adaptées et déployées dans un contexte d’enseignement supérieur en ligne afin d’améliorer la rétention des apprenants ?}

\textbf{Hypothèse.} Nous postulons que les modèles d’IA issus d’autres domaines, s’ils sont adaptés aux spécificités pédagogiques, culturelles et techniques de l’enseignement supérieur en Afrique, peuvent contribuer significativement à réduire le taux d’abandon et à renforcer l’engagement des apprenants.

Nous commencerons par détailler notre méthodologie de revue, avant d’examiner les applications existantes dans le domaine éducatif, puis les contributions d’autres secteurs. Enfin, nous proposerons une expérimentation locale à l’Université Numérique Cheikh Hamidou Kane (UN-CHK), accompagnée d’une réflexion éthique et technique.

\section*{Cadre conceptuel}
L’étude de la rétention des apprenants dans l’enseignement supérieur s’appuie sur plusieurs modèles théoriques qui permettent de comprendre les mécanismes de persévérance, d’engagement et d’abandon.

\subsection*{Le modèle de Tinto}
Le modèle de Tinto (1993) identifie trois facteurs clés influençant la rétention : l'intégration académique, l'intégration sociale et les caractéristiques individuelles. Il postule que plus un étudiant est intégré à la vie universitaire, plus il est susceptible de persévérer.

\subsection*{Engagement multidimensionnel}
L’engagement est défini comme un construit multidimensionnel : comportemental (participation), émotionnel (motivation) et cognitif (efforts de compréhension). Ces dimensions sont mobilisables via des technologies IA telles que les tableaux de bord, les chatbots ou les recommandations.

\subsection*{Apprentissage personnalisé}
L’approche constructiviste met l’accent sur un apprentissage centré sur l’apprenant. L’IA peut favoriser cette personnalisation par une analyse continue des performances et des préférences.

\section{Méthodologie de la revue}
Cette revue repose sur une approche comparative interdisciplinaire. La recherche a été menée sur Google Scholar, HAL, IEEE Xplore, ResearchGate et ACM, entre 2015 et 2024.

\textbf{Mots-clés :} "learner retention", "adaptive learning", "churn prediction", "AI in education", etc.

\textbf{Critères d’inclusion :} études utilisant des techniques d’IA pour la rétention, validées expérimentalement, avec données disponibles ou explicitées.

\textbf{Corpus :} 52 articles retenus dont 28 en éducation et 24 en secteurs connexes. Les solutions ont été classées selon : personnalisation, optimisation de parcours, prédiction des performances.

% (Les sections suivantes à intégrer ou compléter ensuite)

\end{document}
